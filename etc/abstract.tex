\chapter*{Abstract}
\label{cha:abtract}
\addcontentsline{toc}{chapter}{Abstract}

Battery-less computing is without a doubt one of the most interesting frontiers
of innovation for the vast world of the Internet of Things, particularly in the context
of wearable devices. This thesis work proposes a novel approach to a connected
personal heart-rate monitor for fitness related applications, with particular emphasis
on the energy harvesting technologies powering the system.\\ The device is to be
worn on the user's chest during training sessions, so it's paramount to properly
choose the energy sources to exploit: heat transfer between the skin and the outside
air can be harvested in the form of thermoelectric energy, employing a series of
thermoelectric generators - also known as Peltier cells - positioned in contact
with the skin, while a small form factor solar cell converts solar radiation into
usable electric energy. These modules need to be connected to specially-designed
supporting circuitry, consisting of mainly ultra-low or low voltage boost converters/voltage
regulators, which enables us to make the most of the usually low inputs. Other sources
were considered and evaluated, kinetic energy in particular, although not fully implemented.\\
Great care was put into selecting readily-available and up-to-date components which
were both easy to include in the design and reasonably energy-efficient. The
system is controlled by a low-power and small form factor micro controller board
based on a Nordic Semiconductors chip, while the heart rate sensing capabilities
are handled by a standalone Maxim Integrated module.\\ The system communicates with
connected devices via BLE (Bluetooth Low Energy), which greatly reduces power
consumption during transmission of the heart rate data.
