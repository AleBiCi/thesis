\chapter*{Abstract}
\label{cha:abtract}
\addcontentsline{toc}{chapter}{Abstract}

Battery-less computing is without a doubt one of the most interesting frontiers
of innovation for the vast world of IoT devices, particularly in the context of wearables.
This thesis work proposes a novel approach to a connected personal heart-rate
monitor for fitness related applications, with particular emphasis on the energy
harvesting technologies powering the system.\\ Since the device is to be worn during
training sessions in a dynamic environment, energy sources were chosen in the
form of solar and thermoelectric energy. Other sources comprised of the appropriate
auxiliary circuitry were considered and evaluated, most of all kinetic energy. \\
Great care was put into selecting readily-available components which were both easy
to include in the implemented design and reasonably energy-efficient. The system
ic controlled by a low-power and small form factor microcontroller board based
on a Nordic Semiconductors chip, while the heart rate sensing capabilities are handled
by a standalone Maxim Integrated module.
